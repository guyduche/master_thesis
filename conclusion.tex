\chapter{Conclusion}
\label{ch:conclusion}
The aim of this project was to develop a software able to align short-reads using \gls{ncbi}\slash\gls{blast}.
Following these objectives, we created two programs:
\begin{itemize}
    \item \fastqtofasta{} to allow the use of short-reads by \gls{blast};
    \item \blastobam{} to extract the results from \gls{blast}, pair them if need be, analyze them and finally output them in a format which will allow for further analyses by downstream software (\gls{sam}).
\end{itemize}

After implementation, we used it on real data and tested if the output was conform to the format specification and if the results were comparable with those obtained with another aligner.
\blastobam{} passed the tests.
A number of upgrades were then issued to make the programs better, able to work with different types of data and more suited to the user's needs.

We then wondered if \gls{blast}--\blastobam{} could be better, in certain circumstances, than the current gold standard, \gls{bwa}.
Both programs were evaluated on different parameters: time spent, number of reads mapped and coverage.
\gls{blast}--\blastobam{} was definitely slower than \gls{bwa} but seems to be better in terms of coverage.
More tests would be necessary to confirm it.

Although \fastqtofasta{} and \blastobam{} are working properly, there is still room for improvement.
For example, nothing was done, yet, concerning chimeric alignments, i.e.~a read of which two different parts have been found to be matched at different positions on the reference.
Example: the first half of the read is mapped at a position on the reference and the second half is mapped at another place on the reference.
Currently, \blastobam{} considers both but will output the one with the lowest score as the best (see subsection~\ref{subsec:thereCanBeOnlyOne}) and the other one as secondary.

As reported by Peter \textsc{Cock} on \href{http://blastedbio.blogspot.fr/2015/07/ncbi-working-on-sam-output-from-blast.html}{http://blastedbio.blogspot.fr}, the \gls{ncbi} is working on a \gls{sam} output for \gls{blast}. It is present in the current version of \gls{blast}\texttt{+} (2.2.31) as an undocumented option (\texttt{-outfmt 15}).
Like blast2sam.pl, it seems to be a converter. As of now, it does not produce a proper header, the read names are changed, there is no pairing and the sequence quality is lost.
However, it seems to work also with protein sequences.
Although \blastobam{} has been developed with genetic data in mind, it might be interesting to adapt it to enable the use of protein sequences.

To make our software available to the community, we pushed all the source code, tests and manual, including the code of this master's thesis on GitHub at \url{https://github.com/guyduche/Blast2Bam} where, to our delight, it seems to have been appreciated.
